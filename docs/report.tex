\documentclass[a4paper,12pt]{report}

\usepackage[italian]{babel}

\title{\textbf{PyChat}\\Progetto di Programmazione di reti\\\textit{Università di Bologna}}
\author{Martin Tomassi; 0001077737\\Francesco Pazzaglia; 0001077423\\Luca Casadei; 0001069237}
\date{\today}

\begin{document}

\maketitle
\tableofcontents

\chapter{Introduzione}
Questa è la documentazione del progetto di Programmazione di Reti relativo alla Traccia 1: \textit{Sistema di Chat Client-Server} in Python. Il progetto prevede la realizzazione di una chat che consente a più utenti di connettersi a un server centrale e comunicare tra loro in una chatroom condivisa.\\
Il progetto è composto da due componenti principali: il \textbf{client} e il \textbf{server}.\\
Il server gestisce le connessioni dei client e invia i messaggi in modalità broadcast all'interno della chat. Il client, invece, permette agli utenti di connettersi al server tramite protocollo TCP, di inviare messaggi e di visualizzare i messaggi ricevuti dalla chatroom degli altri utenti. Quando un client si connette al server, può vedere solo i messaggi inviati dopo che la connessione è stata stabilita; i messaggi precedenti non vengono "salvati" nel server. \\
E' stato scelto di utilizzare il protocollo a livello di trasporto chiamato \textbf{T}ransimission \textbf{C}ontrol \textbf{P}rotocol perché garantisce la consegna affidabile dei dati, quindi nel caso di una chat è fondamentale che i messaggi vengano inviati e ricevuti in maniera corretta senza alcun tipo di perdita.

\chapter{Client}

\chapter{Server}

\end{document}
