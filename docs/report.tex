\documentclass[a4paper,12pt]{report}

\usepackage[italian]{babel}

\title{\textbf{PyChat}\\Progetto di Programmazione di reti\\\textit{Università di Bologna}}
\author{Martin Tomassi; 0001077737\\Francesco Pazzaglia; 0001077423\\Luca Casadei; 0001069237}
\date{\today}

\begin{document}

\maketitle
\tableofcontents

\chapter{Introduzione}
Questa è la documentazione del progetto di Programmazione di Reti relativo alla Traccia 1: \textit{Sistema di Chat Client-Server} in Python. Il progetto prevede la realizzazione di una chat che consente a più utenti di connettersi a un server centrale e comunicare tra loro in una chatroom condivisa.\\
Il progetto è composto da due componenti principali: il \textbf{client} e il \textbf{server}.\\
Il server gestisce le connessioni dei client e invia i messaggi in modalità broadcast all'interno della chat. Il client, invece, permette agli utenti di connettersi al server tramite protocollo TCP, di inviare messaggi e di visualizzare i messaggi ricevuti dalla chatroom degli altri utenti. Quando un client si connette al server, può vedere solo i messaggi inviati dopo che la connessione è stata stabilita; i messaggi precedenti non vengono "salvati" nel server. \\
E' stato scelto di utilizzare il protocollo a livello di trasporto chiamato \textbf{T}ransimission \textbf{C}ontrol \textbf{P}rotocol perché garantisce la consegna affidabile dei dati, quindi nel caso di una chat è fondamentale che i messaggi vengano inviati e ricevuti in maniera corretta senza alcun tipo di perdita.

\chapter{Client}
Il client utilizza un approccio connection-oriented IPv4 attraverso il modulo \textit{socket} di Python per stabilire una connessione con il server della chatroom. Prima di avviare l'applicazione, l'utente deve specificare l'indirizzo IP del server e un nome utente univoco, che servirà a identificarlo all'interno della chatroom. Questi parametri vengono passati come argomenti al momento dell'esecuzione del client.
\\
Una volta avviato, il client crea un socket e tenta di stabilire una connessione con il server utilizzando l'indirizzo IP e la porta specificati. Se la connessione viene stabilita con successo, il client invia il nome utente al server per identificarsi.
\\
Dopo la connessione, il client avvia due thread separati per gestire la trasmissione e la ricezione dei messaggi. 
\\
Il thread di trasmissione consente all'utente di digitare un messaggio dall'input del terminale e inviarlo al server tramite il socket. Prima di inviare il messaggio, viene applicata la codifica UTF-8 per garantire la compatibilità con caratteri speciali come le lettere accentate.
\\
Il thread di ricezione rimane in attesa di messaggi inviati dal server e li decodifica utilizzando sempre la codifica UTF-8 per garantire una corretta visualizzazione. Quando arriva un nuovo messaggio, il client lo visualizza sulla console.
\\
L'interfaccia utente fornisce all'utente le istruzioni necessarie per interagire con l'applicazione, inclusi i comandi per inviare messaggi e le notifiche sullo stato dell'applicazione, ad esempio l'avviso di CTRL-C per terminare l'esecuzione.
\\
Inoltre, il client tiene traccia degli utenti online all'interno della chatroom e offre la possibilità di visualizzare la lista degli utenti connessi in un determinato momento tramite il comando \textit{/list}.
\chapter{Server}

\end{document}
